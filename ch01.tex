\chapter{Computer Programming}

\begin{exercise}
This exercise will let you practice displaying information.  Write a program named \java{AboutMe.java} with a \java{main} method that prints following:

\begin{enumerate}
\item Your name.
\item Your favorite color.
\item Your favorite word, in double quotes.
\item Your favorite poem, at least three lines long (If your favorite poem is longer than ten lines, give only the first three to ten lines.)
\item The author of the poem. The name should be indented four spaces and preceded by a hyphen and a space. If you don't know who wrote the poem, use ``Anonymous'' as the author.
\end{enumerate}

Use complete sentences and blank lines for readability.  To print a blank line, use:

\java{System.out.println();}

The parentheses are required, even though there is nothing between them. If you don't believe me, leave them out and see what the compiler has to say about it!

Here is what the output of my solution looks like:

\begin{stdout}
My name is David Eisenberg.
My favorite color is purple.
My favorite word is "kinkajou".

My favorite poem:

Beneath these high Cathedral stairs
Lie the remains of Susan Pares.
Her name was Wiggs, it was not Pares,
But Pares was put to rhyme with stairs.
    - Edward Lear
\end{stdout}
\end{exercise}
