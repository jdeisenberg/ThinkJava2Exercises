\chapter*{Preface}

\markboth{PREFACE}{PREFACE}
\addcontentsline{toc}{chapter}{Preface}

{\it Exercises for Think Java} is a companion volume to {\it Think Java 2nd Edition} by Allen B. Downey and Chris Mayfield 
\url{https://greenteapress.com/wp/think-java-2e/}

Evergreen Valley College, where I teach, has been encouraging faculty to use low-cost or open source textbooks. In this spirit, the Computer Science department decided to use {\it Think Java} for our introductory Computer Science course. While the book met most of our needs, we wanted more exercises for students to practice programming. There were also some topics that we cover in our course that were not in the book. Thus, we decided to add some material and more exercises.

Instead of editing the original book, which is open source, I have decided to make this a separate book for two reasons:

\begin{enumerate}
\item Inserting many exercises, some of which require a great deal of explanation, breaks up the flow of the material. 
\item The modified book would no longer match the print version, which some students prefer to use.
\end{enumerate}

\section*{Using the Code Examples}
\label{code}

Most of the code examples in this book are available from a Git repository at \url{https://github.com/jdeisenberg/ThinkJava2ExCode}.
Git is a ``version control system'' that allows you to keep track of the files that make up a project.
A collection of files under Git's control is called a ``repository''.

\index{repository}
\index{GitHub}

GitHub is a hosting service that provides storage for Git repositories and a convenient web interface.
It provides several ways to work with the code:

\begin{itemize}

\item You can create a copy of the repository on GitHub by clicking the {\sf Fork} button.
If you don't already have a GitHub account, you'll need to create one.
After forking, you'll have your own repository on GitHub that you can use to keep track of code you write.
Then you can ``clone'' the repository, which downloads a copy of the files to your computer.

\item Alternatively, you could clone the original repository without forking.
If you choose this option, you don't need a GitHub account, but you won't be able to save your changes on GitHub.

\item If you don't want to use Git at all, you can download the code in a ZIP archive using the {\sf Clone} button on the GitHub page.

\end{itemize}

After you clone the repository or unzip the ZIP file, you should have a directory named {\it ThinkJava2ExCode} with a subdirectory for each chapter in the book that contains sample code.

%\medskip

If you have additional comments or ideas about the text, please send them to: \href{mailto:david.eisenberg@evc.edu}{\tt david.eisenberg@evc.edu}.

\section*{Acknowledgement}
Thanks to Prof. Jack Ho, who provided some exercises for the book.

\rule{\textwidth}{0.1mm}

\hfill J David Eisenberg

\hfill July 2021
