\chapter{Loops and Strings}

\begin{exercise}
Write a program that asks the user for a starting amount of money, an annual interest rate as a percent, and a number of years. Use this information to print a table that shows the balance with accumulated compound interest. Use a \java{for} loop. Here is what your output might look like:

\begin{stdout}
Enter starting amount: $100
Enter annual percent interest: 5.3
Enter number of years: 4
Year  Balance
0     $100.00
1     $105.30
2     $110.88
3     $116.76
4     $122.95
\end{stdout}

\end{exercise}
\begin{exercise}
When you go to the store, the clerk doesn't know in advance how many items you want to buy. Instead, they keep totaling the items until they see one of those plastic dividers to indicate that your order is complete.

\index{sentinel value}
\index{value!sentinel}
Write a program that repeatedly asks the user for the price of an item until they enter a zero for the price. This is the digital equivalent of the plastic divider. Its technical name is a {\em sentintel value}. As the user enters prices, keep track of the total number of items and sum of the prices.  After encountering the sentinel value, your program will print the number of items purchased, the subtotal, the tax (at a rate of 6.5\%), and the grand total. Here is what the program might look like. Note that it does not allow (or count) negative prices. {\em Hint}: Use a \java{while} loop.

\begin{stdout}
Enter price, or 0 when finished: $3.50    
Enter price, or 0 when finished: $-2
Prices can not be negative.
Enter price, or 0 when finished: $5.99
Enter price, or 0 when finished: $4.83
Enter price, or 0 when finished: $0

Number of items: 3
Subtotal:  $   14.32
Tax:       $    0.93
Total:     $   15.25
\end{stdout}
\end{exercise}


\begin{exercise}
Write a program that repeatedly asks the user to enter a sentence (until they press only ENTER). For each sentence, tell how many vowels, consonants, digits, and ``other'' characters are in the sentence.  For this exercise, presume that the letter ``y'' is a vowel. Continue asking for sentences the user presses only the ENTER key for input. When that happens, the \java{String} that you read will equal the empty string \java{""}. You can test this condition to end the loop.

Sample output:

\begin{stdout}
Input a sentence (just ENTER to quit): 4 score & 7 years ago.
Vowels:   7  Consonants:  6
Digits:   2  Others:      7

Input a sentence (just ENTER to quit): 2*x=17
Vowels:   0  Consonants:  1
Digits:   3  Others:      2

Input a sentence (just ENTER to quit): The Quick Brown Fox
Vowels:   5  Consonants: 11
Digits:   0  Others:      3

Input a sentence (just ENTER to quit):
\end{stdout}

{\em Hint}: Make \java{String} variables with contents such as \java{"aeiouy"}, \java{"bcdfghjklmnpqrstvwxz"}, etc. and use \java{indexOf} to determine whether a character belongs to that group of characters.

\end{exercise}

\begin{exercise}
Write a method called \java{switchOrder}, which takes a \java{String} parameter with a person's name in the form \java{"first middle last"} and returns the name in the form \java{"last, first middle"}.  Your code should work for people with only a single name, people with no middle name, and people with several middle names.
Use a \java{while} loop with the \java{indexOf} and \java{substring} methods to do this exercise. (You could do it more easily with the \java{lastIndexOf} method, but we want you to get practice using loops.)

Write a \java{main} method that uses a \java{while} loop to repeatedly ask the user for a name and then calls the \java{switchOrder} method and prints the result. Repeat until the user presses just ENTER. Here is some sample output:

\begin{stdout}
Input a name (just ENTER to quit): Grace Murray Hopper
Hopper, Grace Murray

Input a name (just ENTER to quit): Donald Knuth   
Knuth, Donald

Input a name (just ENTER to quit): Prince
Prince

Input a name (just ENTER to quit): C. Anthony Richard Hoare
Hoare, C. Anthony Richard

Input a name (just ENTER to quit): 
\end{stdout}

\end{exercise}

\begin{exercise}
Write a program named{\it PhoneWord.java} that prompts the user for a ``phone word,'' an alphabetic mnemonic for a phone number. Then, print out the phone number corresponding to that sequence.

Here is how your program must translate letters to numbers:

\begin{tabular}{|l|l|}
\hline
ABC & 2 \\ \hline
DEF & 3 \\ \hline
GHI & 4 \\ \hline
JKL & 5 \\ \hline
MNO & 6 \\ \hline
PQRS & 7 \\ \hline
TUV & 8  \\ \hline
WXYZ & 9  \\ \hline
\end{tabular}

Keep digits as digits (see the second example output). Don't forget about zero!

You must accept letters in either upper or lower case. If the phone word translates to more than seven digits, keep only the first seven. If the phone word translates to fewer than seven digits, print an error message. Ignore any characters other than a letter or digit.

Here is an example of several runs of the program:

\begin{stdout}
Enter a phone word: warbler
The number is 9272537.

Enter a phone word: GOOD4U2
The number is 4663482.

Enter a phone word: OMG
Your phone word is not long enough for a phone number.

Enter a phone word: GREAT DEALS
The number is 4732833.

Enter a phone word: got-food?
The number is 4683663.

\end{stdout}

{\em Hint}: Do {\em not} check the input string to see if its length is greater than or equal to seven. The string 
\java{"C-A-T-S!"} is eight characters long, but there are only four letters, so it will not translate to a valid phone word.

Instead, convert all the letters in the string to digits, no matter how many or how few, and {\em then} check the length of the result to see if it is seven characters or more.

Extra challenge: Print the phone number with a hyphen; for example: 473-2833.

\end{exercise}
