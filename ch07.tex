\chapter{Arrays and References}

\begin{exercise}
\label{ex:standardize}
Write the following methods, each of which takes an array of \java{double} values as a parameter:

\begin{itemize}
\item The \java{mean} method returns the arithmetic mean (average) of the array elements.

\item The \java{stdev} method returns the standard deviation of the array elements, using this formula
\begin{equation*}
s = \sqrt {{n \sum{x_i} ^2 - \left(\sum{x_i}\right)^2} \over {n (n - 1)}}
\end{equation*}

where $n$ is the number of items in the array, $\sum{x_i} ^2$ is the sum of the squares of the individual items, and $\left(\sum{x_i}\right)^2$ is the square of the total of all the items in the array.

\item The \java{standardize} returns a new array of the values converted to {\em standard scores}. First, calculate the mean $m$ and standard deviation $s$ of the entries in the original array. Each entry in the result array will be \java{(arr[i] - m) / s}. This method must call the \java{mean} and \java{stdev} methods.

Write a \java{main} method to test the \java{standardize} method. For example, an array with values
\java{\{47.0, 11.0, 10.0, 66.0, 8.5\}} will generate these standard scores: \java{\{0.700, -0.662, -0.700, 1.418, -0.756\}} (to three decimal places). 

\end{itemize}
\end{exercise}

\begin{exercise}
\label{ex:stringReversal}

Unlike the string reversal method shown in {\em Think Java}, where a program built a new string that had its characters in the reverse order of the original, in this exercise you will write a \java{void} method named \java{reverseInPlace} that takes an array of integers as its parameter and reverses the order of the items in the array--in place. In other words, you won't create a new array; you will change the order of the items within the array itself.

Write a \java{main} method that will test the method by reversing and displaying:

\begin{enumerate}
\item An empty array with no elements
\item An array with one element
\item An array with an even number of elements (more than two elements)
\item An array with an odd number of elements (more than three elements)
\end{enumerate}

To make your job easier, you might want to create a ``helper method'' named \java{reverseAndDisplay} that prints the array, calls \java{reverseInPlace} and prints the array (which should now be reversed).
\end{exercise}

\begin{exercise}
\label{ex:vectors}

You can use an array of three \java{double} values to represent a {\em vector} in three-dimensional space. The first element represents the vector's $x$-component, the second its $y$ component, and the third its $z$ component, written as $(x, y, z)$. Write a program that has these methods, each of which takes two three-element arrays as parameters:
\begin{itemize}
\item \java{add}: Returns a vector (an array of length three) representing the sum of the parameters. The sum of $(x_1, y_1, z_1)$ and $(x_2, y_2, z_2)$ is the vector $(x_1 + x_2, y_1 + y_2, z_1 + z_2)$
\item \java{dotProduct}: Returns the {\em dot product} of the parameter vectors. The dot product of  $(x_1, y_1, z_1)$ and $(x_2, y_2, z_2)$ is calculated as $x_1\cdot x_2 + y_1\cdot y_2 + z_1\cdot z_2$ 
\item \java{distance}: Returns the distance between the vectors $(x_1, y_1, z_1)$ and $(x_2, y_2, z_2)$ using the formula $\sqrt{(x_1 - x_2)^2 + (y_1 - y_2)^2 + (z_1 - z_2)^2}$
\end{itemize}

The \java{main} method will ask the user to enter two vectors and then display the sum, dot product, and distance between the vectors. To avoid repetitious code, you might want to write a \java{getVector} method that has a prompt and a \java{Scanner} as its parameters. This method will prompt the user for the three vector components and return an array of three \java{double} values.

Here is an example of what the program might look like:

\begin{stdout}
Enter components of the first vector,
separated by spaces: 2 1.5 4.1
Enter components of the second vector,
separated by spaces: 7 3.5 1.2
Sum of vectors: (9.000, 5.000, 5.300)
Distance between vectors: 6.116
Dot product of vectors: 24.170
\end{stdout}

\end{exercise}

\begin{exercise}
In this exercise, you will find the correlation between the elements in two arrays of equal length.
Your program will ask the user for the number of entries in each array, read them in, and see how good a linear relationship they have to each other by calculating the {\em correlation coefficient}. This is a number that ranges from 1.0 (the two arrays are perfectly related to each other) to -1.0 (the arrays are perfectly inversely correlated to one another---as an example, think of the heights of the two ends of a seesaw in motion). A correlation of zero means the values in the two arrays have no linear relationship.

Write a method named \java{correlation} that takes two arrays of double and returns the correlation coefficient $r$ according to this formula:

\begin{equation*}
%r = {{\sum x_iy_i - \left(\sum x_i \sum y_i \over n\right)} \over {(n - 1) s_x s_y}}
r = {{n \sum x_iy_i - (\sum x_i)(\sum y_i)} \over {\sqrt {\left[n \sum x_i^2 - (\sum x_i)^2\right]\left[n \sum y_i^2 - (\sum y_i)^2\right]}}}
\end{equation*}

Here is an example of what the output might look like. The data are people's height in centimeters for the first array and their weight in kilograms for the second array.

\begin{stdout}
Enter number of elements in each array: 5
Enter the 5 elements in the first array: 178 176 160 180 186
Enter the 5 elements in the second array: 84.2 77.3 60 65.5 82.1
The correlation coefficient is 0.710.
\end{stdout}

\end{exercise}

\begin{exercise}
Write a method called \java{separate} which takes an array of integers as its argument and returns a new array that consists of all odd elements in the first array followed by all the even numbers in the first array, in the same relative order.

This method will require two passes (traversals) through the array.

Then, write a \java{main} method to test the \java{separate} method. Here is what it
might look like:

\begin{stdout}
Enter number of items in array: 6
Enter the items, separated by spaces: 10 47 11 66 5 98
Array with odd, then even elements: [47, 11, 5, 10, 66, 98]
\end{stdout}

You may use the \java{java.utils.Arrays.toString} method to print the result array. Make sure you test your program with an array that contains only odd numbers and again with only even numbers.

\end{exercise}

